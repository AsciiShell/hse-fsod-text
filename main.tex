%!TEX TS-program = xelatex
\documentclass[a4paper,14pt]{article}
\usepackage[utf8]{inputenc}
\usepackage[T1]{fontenc}

%%% Работа с русским языком
\usepackage[english,russian]{babel}   %% загружает пакет многоязыковой вёрстки
\usepackage{fontspec}      %% подготавливает загрузку шрифтов Open Type, True Type и др.
\defaultfontfeatures{Ligatures={TeX},Renderer=Basic}  %% свойства шрифтов по умолчанию
\setmainfont[Ligatures={TeX,Historic}]{Times New Roman} %% задаёт основной шрифт документа
\setsansfont{Comic Sans MS}                    %% задаёт шрифт без засечек
\setmonofont{Courier New}
\usepackage{indentfirst}
\frenchspacing

\renewcommand{\epsilon}{\ensuremath{\varepsilon}}
\renewcommand{\phi}{\ensuremath{\varphi}}
\renewcommand{\kappa}{\ensuremath{\varkappa}}
\renewcommand{\le}{\ensuremath{\leqslant}}
\renewcommand{\leq}{\ensuremath{\leqslant}}
\renewcommand{\ge}{\ensuremath{\geqslant}}
\renewcommand{\geq}{\ensuremath{\geqslant}}
\renewcommand{\emptyset}{\varnothing}

%%% Дополнительная работа с математикой
\usepackage{amsmath,amsfonts,amssymb,amsthm,mathtools} % AMS
\usepackage{icomma} % "Умная" запятая: $0,2$ --- число, $0, 2$ --- перечисление

%% Номера формул
%\mathtoolsset{showonlyrefs=true} % Показывать номера только у тех формул, на которые есть \eqref{} в тексте.
%\usepackage{leqno} % Нумерация формул слева	

%% Перенос знаков в формулах (по Львовскому)
\newcommand*{\hm}[1]{#1\nobreak\discretionary{}
{\hbox{$\mathsurround=0pt #1$}}{}}

%%% Работа с картинками
\usepackage{graphicx}  % Для вставки рисунков
\graphicspath{{images/}}  % папки с картинками
\setlength\fboxsep{3pt} % Отступ рамки \fbox{} от рисунка
\setlength\fboxrule{1pt} % Толщина линий рамки \fbox{}
\usepackage{wrapfig} % Обтекание рисунков текстом

%%% Работа с таблицами
\usepackage{array,tabularx,tabulary,booktabs} % Дополнительная работа с таблицами
\usepackage{longtable}  % Длинные таблицы
\usepackage{multirow} % Слияние строк в таблице
\usepackage{float}% http://ctan.org/pkg/float

%%% Программирование
\usepackage{etoolbox} % логические операторы


%%% Страница
\usepackage{extsizes} % Возможность сделать 14-й шрифт
\usepackage{geometry} % Простой способ задавать поля
\geometry{top=20mm}
\geometry{bottom=20mm}
\geometry{left=20mm}
\geometry{right=10mm}
%
%\usepackage{fancyhdr} % Колонтитулы
% 	\pagestyle{fancy}
%\renewcommand{\headrulewidth}{0pt}  % Толщина линейки, отчеркивающей верхний колонтитул
% 	\lfoot{Нижний левый}
% 	\rfoot{Нижний правый}
% 	\rhead{Верхний правый}
% 	\chead{Верхний в центре}
% 	\lhead{Верхний левый}
%	\cfoot{Нижний в центре} % По умолчанию здесь номер страницы

\usepackage{setspace} % Интерлиньяж
\onehalfspacing % Интерлиньяж 1.5
%\doublespacing % Интерлиньяж 2
%\singlespacing % Интерлиньяж 1

\usepackage{lastpage} % Узнать, сколько всего страниц в документе.

\usepackage{soul} % Модификаторы начертания

\usepackage[hyphens]{url}
\usepackage{hyperref}
\usepackage[usenames,dvipsnames,svgnames,table,rgb]{xcolor}
\hypersetup{                % Гиперссылки
    unicode=true,           % русские буквы в раздела PDF
    pdftitle={Детекция объектов по нескольким примерам без дообучения},   % Заголовок
    pdfauthor={Самоделкина М.В., Подчезерцев А.Е.},      % Автор
    pdfsubject={Детекция объектов по нескольким примерам без дообучения},      % Тема
    pdfcreator={Самоделкина М.В., Подчезерцев А.Е.}, % Создатель
    pdfproducer={Самоделкина М.В., Подчезерцев А.Е.}, % Производитель
%    pdfkeywords={keyword1} {key2} {key3}, % Ключевые слова
    colorlinks=true,        % false: ссылки в рамках; true: цветные ссылки
    linkcolor=black,          % внутренние ссылки
    citecolor=black,        % на библиографию
    filecolor=magenta,      % на файлы
    urlcolor=black           % на URL
}
\makeatletter
\def\@biblabel#1{#1. }
\makeatother
\usepackage{cite} % Работа с библиографией
%\usepackage[superscript]{cite} % Ссылки в верхних индексах
%\usepackage[nocompress]{cite} % 
\usepackage{csquotes} % Еще инструменты для ссылок

\usepackage{multicol} % Несколько колонок

\usepackage{tikz} % Работа с графикой
\usepackage{pgfplots}
\usepackage{pgfplotstable}

\usepackage{ dsfont }

\newcommand{\imref}[1]{рис.~\ref{#1}}

\usepackage{multirow}
\usepackage{spreadtab}
\newcolumntype{K}[1]{@{}>{\centering\arraybackslash}p{#1cm}@{}}


\usepackage{xparse}
\usepackage{fancyvrb}

\RecustomVerbatimCommand{\VerbatimInput}{VerbatimInput}
{
    fontsize=\footnotesize
}

\newcolumntype{?}[1]{!{\vrule width #1}}

\usepackage{tocloft}
\renewcommand{\cftsecleader}{\cftdotfill{\cftdotsep}}

\usepackage{pdfpages}

\usepackage{rotating}

\usepackage{pdflscape}

\usepackage{ragged2e}
\usepackage{microtype}

\usepackage{subcaption}

% Выравнивание по ширине без переносов слов
\justifying
\sloppy
\tolerance=500
\hyphenpenalty=10000
\emergencystretch=3em

% Подогнать таблицу под ширину страницы
\usepackage{adjustbox}

\usepackage{titlesec}

% ГОСТ заголовки таблицы
\usepackage[font=small]{caption}

\captionsetup[figure]{justification=centering,labelsep=period} % Картинки по центру, с точкой после рис

\DeclareCaptionLabelFormat{rightline}{\rightline{\bothIfFirst{#1}{ }#2}}
%\captionsetup[table]{justification=centering,labelformat=rightline,labelsep=newline}
\captionsetup[table]{justification=raggedleft,labelformat=rightline,labelsep=newline}

\newcommand{\tablecaption}[1]{\addtocounter{table}{1}\small \begin{flushright}
                                                                \tablename \ \thetable
\end{flushright}%
    \begin{center}
        #1
    \end{center}}

\usepackage{enumerate}

%сноска с новым номером на кждоый странице
\usepackage[perpage]{footmisc}
\usepackage[fixlanguage]{babelbib}
\usepackage{lastpage}
\begin{document}

	\begin{titlepage}
    \begin{center}
        ПРАВИТЕЛЬСТВО РОССИЙСКОЙ ФЕДЕРАЦИИ \\
        ФЕДЕРАЛЬНОЕ ГОСУДАРСТВЕННОЕ АВТОНОМНОЕ \\
        ОБРАЗОВАТЕЛЬНОЕ УЧРЕЖДЕНИЕ ВЫСШЕГО ОБРАЗОВАНИЯ\\
        «НАЦИОНАЛЬНЫЙ ИССЛЕДОВАТЕЛЬСКИЙ УНИВЕРСИТЕТ\\
        «ВЫСШАЯ ШКОЛА ЭКОНОМИКИ»
    \end{center}

    \begin{center}
        \textbf{Факультет компьютерных наук}

        \textbf{Образовательная программа <<Финансовые технологии и анализ данных>>}

%        \vspace{2ex}

    \end{center}
    \vspace{1ex}

    \begin{center}
        \textbf{Курсовая работа \\
            <<Детекция объектов по нескольким примерам без дообучения>>
        }
    \end{center}

    \vspace{2ex}
    \vfill

    \vspace{2ex}

    \begin{flushright}
        \textbf{Выполнили:}

        \vspace{2ex}

        Студенты группы мФТиАД21

        \vspace{2ex}

        Самоделкина Мария Владимировна

        Подчезерцев Алексей Евгеньевич

        \vspace{2ex}
        \textbf{Руководитель:}

        Приглашенный преподаватель

        Озерин Алексей Юрьевич

    \end{flushright}

    \vspace{5ex}
    \begin{center}
        Москва \the\year \, г.
    \end{center}

\end{titlepage}
\addtocounter{page}{1}

    \section*{\normalsize \hfill Аннотация \hfill}

    ...

    \sloppy
    \newpage

    \section*{\normalsize \hfill Abstract \hfill}

    ...

    \newpage

    \tableofcontents
    \pagebreak

    \section*{Введение}
    \addcontentsline{toc}{section}{\protect\numberline{}Введение}


    \newpage


    \section{Обзор литературы}
    
    \subsection{Основные понятия}
    
    Задача детекции объектов на изображении включает в себя несколько подзадач:
    \begin{enumerate}
    	[1)]
    	\itemsep0em
    	\item определение прямоугольной области, ограничивающей объект (bounding box, bbox); 
    	\item классификация выделенной области; 
    \end{enumerate}
	В результате работы модели, решающей задачу поиска объектов, получаются области или регионы изображения (object proposals) в которых может находиться объект, а также определена вероятность принадлежности этого объекта к определенному классу.
	
	Дополнительно могут решаться задачи нахождения ключевых точек объекта и сегментации - отделения объекта от фона.
	
	Наиболее популярные датасеты для обучения модели детекции объектов: MS~COCO~\cite{COCO}, LVIS~\cite{LVIS}, Objects365~\cite{Objects365}.
	
	Эмбеддинг (embedding) - скрытое векторное представление объекта (текстовых, графические, табличных и аудио данных) фиксированной размерности.
	
	Карта признаков (feature map) - векторное представление изображения нефиксированной размерности, где элемент карты содержит информацию о некоторой области изображения.
	
	R-CNN
	
	Yolo
	
	CLIP~\cite{CLIP}
	
	\subsection{Детекция объектов по произвольному текстовому запросу}
	
	В статье~\cite{ViLD} предложен метод ViLD - Vision and Language knowledge Distillation, позволяющий находить объекты с произвольным текстовым описанием. 
	На этапе обучения рассчитываются текстовые эмбеддинги описаний категорий и эмбеддинги регионов изображения с использованием предобученной модели (модель-учитель). 
	Обучается модель детекции объектов R-CNN (модель-ученик), которая дополнительно минимизирует расстояние между эмбеддингом региона предобученной модели-учителя и эмбеддингами региона модели-ученика. 
	Таким образом, происходит извлечение знаний (дистилляция) из модели-учителя. 
	Во время применения модели-ученика предобученная модель-учитель используется только для получения текстовых эмбеддингов, а эмбеддинги для регионов получается из модели-ученика. 
	Итоговые регионы получаются в результате ранжирования оценок, полученных как скалярное произведение текстового эмбеддинга модели-учителя и эмбеддинга региона модели-ученика.
	 
	В качестве модели-учителя в статье используется модель CLIP. 
	Модель обучена на датасете LVIS. В работе отмечается, что модель может применяться и к другим датасетам (COCO, Objects365) без дообучения.
	
	Преимуществом описанного подхода является использование одной модели как для выделения регионов, так и для получения их эмбеддингов, что ускоряет время применения, однако точность при таком подходе понижается.
	
	Полученную модель предлагается использовать следующим образом: для изображения подается набор текстовых описаний, выбираются регионы, получаются их эмбеддинги, в результате отбирается заданное количество регионов с минимальным скалярным произведением. 
	
	К недостаткам предлагаемого подхода можно отнести поиск текстовых описаний только на одном изображении, то есть при отсутствии объектов, релевантных набору текстовых описаний, алгоритм  выведет требуемое количество регионов (для нерелевантных текстовых описаний скалярное произведение также может быть высокое).
	
    \subsection{Цель и задачи исследования}

    Задачи:
    \begin{enumerate}
        [1)]
        \itemsep0em
        \item йцу

    \end{enumerate}

    \subsection{Выводы к главе \thesection}
    \begin{enumerate}
        [1)]
        \itemsep0em
        \item йцу
    \end{enumerate}

    \newpage


    \section{Теоретическая часть}

    \subsection{Выводы к главе \thesection}
    \begin{enumerate}
        [1)]
        \itemsep0em
        \item йцу
    \end{enumerate}

    \newpage


    \section{Практическая часть}

    \subsection{Выводы к главе \thesection}



    \newpage


    \section{Заключение}

    В работе изучена...

    \newpage
    \renewcommand{\refname}{{\normalsize \hfill Список использованных источников \hfill}}
%    \bibliographystyle{unsrt}
    \selectbiblanguage{russian}
    \bibliographystyle{BibTeX-Styles/ugost2008mod}
    \bibliography{main}
    \newpage

\end{document}