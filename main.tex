%!TEX TS-program = xelatex
\documentclass[a4paper,14pt]{article}
\usepackage[utf8]{inputenc}
\usepackage[T1]{fontenc}

\input{data/preambular.tex}
\usepackage[fixlanguage]{babelbib}
\usepackage{lastpage}
\begin{document}

	\begin{titlepage}
    \begin{center}
        ПРАВИТЕЛЬСТВО РОССИЙСКОЙ ФЕДЕРАЦИИ \\
        ФЕДЕРАЛЬНОЕ ГОСУДАРСТВЕННОЕ АВТОНОМНОЕ \\
        ОБРАЗОВАТЕЛЬНОЕ УЧРЕЖДЕНИЕ ВЫСШЕГО ОБРАЗОВАНИЯ\\
        «НАЦИОНАЛЬНЫЙ ИССЛЕДОВАТЕЛЬСКИЙ УНИВЕРСИТЕТ\\
        «ВЫСШАЯ ШКОЛА ЭКОНОМИКИ»
    \end{center}

    \begin{center}
        \textbf{Факультет компьютерных наук}

        \textbf{Образовательная программа <<Финансовые технологии и анализ данных>>}

%        \vspace{2ex}

    \end{center}
    \vspace{1ex}

    \begin{center}
        \textbf{Курсовая работа \\
            <<Детекция объектов по нескольким примерам без дообучения>>
        }
    \end{center}

    \vspace{2ex}
    \vfill

    \vspace{2ex}

    \begin{flushright}
        \textbf{Выполнили:}

        \vspace{2ex}

        Студенты группы мФТиАД21

        \vspace{2ex}

        Самоделкина Мария Владимировна

        Подчезерцев Алексей Евгеньевич



    \end{flushright}

    \vspace{5ex}
    \begin{center}
        Москва \the\year \, г.
    \end{center}

\end{titlepage}
\addtocounter{page}{1}

    \section*{\normalsize \hfill Аннотация \hfill}

    ...

    \sloppy
    \newpage

    \section*{\normalsize \hfill Abstract \hfill}

    ...

    \newpage

    \tableofcontents
    \pagebreak

    \section*{Введение}
    \addcontentsline{toc}{section}{\protect\numberline{}Введение}


    \newpage


    \section{Обзор литературы}
    
    \subsection{Основные понятия}
    
    Задача детекции объектов на изображении включает в себя несколько подзадач:
    \begin{enumerate}
    	[1)]
    	\itemsep0em
    	\item определение прямоугольной области, ограничивающей объект (bounding box, bbox); 
    	\item классификация выделенной области; 
    \end{enumerate}
	В результате работы модели, решающей задачу поиска объектов, получаются области или регионы изображения (object proposals) в которых может находиться объект, а также определена вероятность принадлежности этого объекта к определенному классу.
	
	Дополнительно могут решаться задачи нахождения ключевых точек объекта и сегментации - отделения объекта от фона.
	
	Наиболее популярные датасеты для обучения модели детекции объектов: MS~COCO~\cite{COCO}, LVIS~\cite{LVIS}, Objects365~\cite{Objects365}.
	
	Эмбеддинг (embedding) - скрытое векторное представление объекта (текстовых, графические, табличных и аудио данных) фиксированной размерности.
	
	Карта признаков (feature map) - векторное представление изображения нефиксированной размерности, где элемент карты содержит информацию о некоторой области изображения.
	
	Mask R-CNN
	
	Yolo
	
	CLIP (Contrastive Language–Image Pre-training)~\cite{CLIP} - модель, позволяющая переводить текстовые описания и изображения в единое векторное пространство. 
	CLIP предоставляет возможность решать задачу детекции объектов без дообучения для произвольных классов.
	В основе модели лежат Text Encoder и Image Encoder для преобразования текста и изображения в векторные представления, косинусное расстояние между которыми максимизируется для релевантных пар текста и изображения и минимизируется для нерелевантных.
	
	\subsection{Детекция объектов по произвольному текстовому запросу}
	
	В статье~\cite{ViLD} предложен метод ViLD - Vision and Language knowledge Distillation, позволяющий находить объекты с произвольным текстовым описанием. 
	На этапе обучения рассчитываются текстовые эмбеддинги описаний категорий и эмбеддинги регионов изображения с использованием предобученной модели (модель-учитель). 
	Обучается модель детекции объектов Mask R-CNN (модель-ученик), которая дополнительно максимизирует расстояние между эмбеддингом региона предобученной модели-учителя и эмбеддингами региона модели-ученика. 
	Таким образом, происходит извлечение знаний (дистилляция) из модели-учителя. 
	Во время применения модели-ученика предобученная модель-учитель используется только для получения текстовых эмбеддингов, а эмбеддинги для регионов получается из модели-ученика. 
	Итоговые регионы получаются в результате ранжирования оценок, полученных как скалярное произведение текстового эмбеддинга модели-учителя и эмбеддинга региона модели-ученика.
	 
	В качестве модели-учителя в статье используется модель CLIP. 
	Модель обучена на датасете LVIS. В работе отмечается, что модель может применяться и к другим датасетам (COCO, Objects365) без дообучения.
	
	Преимуществом описанного подхода является использование одной модели как для выделения регионов, так и для получения их эмбеддингов, что ускоряет время применения, однако точность при таком подходе понижается.
	
	Полученную модель предлагается использовать следующим образом: для изображения подается набор текстовых описаний, выбираются регионы, получаются их эмбеддинги, в результате отбирается заданное количество регионов с минимальным скалярным произведением. 
	
	К недостаткам предлагаемого подхода можно отнести поиск текстовых описаний только на одном изображении, то есть при отсутствии объектов, релевантных набору текстовых описаний, алгоритм  выведет требуемое количество регионов (для нерелевантных текстовых описаний скалярное произведение также может быть высокое).
	
    \subsection{Цель и задачи исследования}

    Задачи:
    \begin{enumerate}
        [1)]
        \itemsep0em
        \item йцу

    \end{enumerate}

    \subsection{Выводы к главе \thesection}
    \begin{enumerate}
        [1)]
        \itemsep0em
        \item йцу
    \end{enumerate}

    \newpage


    \section{Теоретическая часть}

    \subsection{Выводы к главе \thesection}
    \begin{enumerate}
        [1)]
        \itemsep0em
        \item йцу
    \end{enumerate}

    \newpage


    \section{Практическая часть}

    \subsection{Выводы к главе \thesection}



    \newpage


    \section{Заключение}

    В работе изучена...

    \newpage
    \renewcommand{\refname}{{\normalsize \hfill Список использованных источников \hfill}}
%    \bibliographystyle{unsrt}
    \selectbiblanguage{russian}
    \bibliographystyle{BibTeX-Styles/ugost2008mod}
    \bibliography{main}
    \newpage

\end{document}